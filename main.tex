
%%%%%%%%%%%%%%%%%%%%%%%%%%%%%%%%%%%%%%%%%
% Developer CV
% LaTeX Template
% Version 1.1 (February 24, 2025)
%
% This template originates from:
% https://www.LaTeXTemplates.com
%
% Authors:
% Jan Vorisek (jan@vorisek.me)
% Based on a template by Jan Küster (info@jankuester.com)
% Modified for LaTeX Templates by Vel (vel@LaTeXTemplates.com)
%
% License:
% The MIT License (see included LICENSE file)
%
%%%%%%%%%%%%%%%%%%%%%%%%%%%%%%%%%%%%%%%%%

%----------------------------------------------------------------------------------------
%	PACKAGES AND OTHER DOCUMENT CONFIGURATIONS
%----------------------------------------------------------------------------------------

\documentclass[9pt]{developercv} % Default font size, values from 8-12pt are recommended

\usepackage{multibbl}
%----------------------------------------------------------------------------------------

\begin{document}

%----------------------------------------------------------------------------------------
%	TITLE AND CONTACT INFORMATION
%----------------------------------------------------------------------------------------

\begin{minipage}[t]{0.45\textwidth} % Left column with your name and title, change the width as needed
	\vspace{-\baselineskip} % Required for vertically aligning minipages
	
	\colorbox{black}{{\HUGE\textcolor{white}{\textbf{\MakeUppercase{Edoardo}}}}} % First name
	
	\colorbox{black}{{\HUGE\textcolor{white}{\textbf{\MakeUppercase{Pedicillo}}}}} % Last name
	
	\vspace{6pt} % Vertical whitespace
	
	{\huge PhD Candidate in Physics} % Career or current job title
\end{minipage}
\hfill % Automatic horizontal whitespace
\begin{minipage}[t]{0.27\textwidth} % Center column with the first column of icons
	\vspace{-\baselineskip} % Required for vertically aligning minipages
	
	\icon{MapMarker}{10}{University of Milan, Italy}\\
	\icon{MapMarker}{10}{TII, Abu Dhabi}\\
	\icon{At}{10}{\href{mailto:edoardo.pedicillo@tii.ae}{edoardo.pedicillo@tii.ae}}\\	
\end{minipage}
\hfill % Automatic horizontal whitespace
\begin{minipage}[t]{0.27\textwidth} % Right column with the first column of icons
	\vspace{-\baselineskip} % Required for vertically aligning minipages
	
	\icon{Github}{10}{\href{https://github.com/Edoardo-Pedicillo}{Edoardo-Pedicillo}}\\
	\icon{Linkedin}{10}{\href{https://www.linkedin.com/in/edoardo-pedicillo-00278426a/}{edoardo-pedicillo}}\\
	\icon{GraduationCap}{10}{\href{https://scholar.google.com/citations?user=BdR0ITEAAAAJ&hl=en}{Edoardo Pedicillo}}\\
\end{minipage}

\vspace{0.5cm} % Vertical whitespace

%----------------------------------------------------------------------------------------
%	INTRODUCTION, SKILLS AND TECHNOLOGIES
%----------------------------------------------------------------------------------------

\cvsect{Who Am I?}

\begin{minipage}[t]{0.4\textwidth} % Left column with the introduction text, change the width as needed
	\vspace{-\baselineskip} % Required for vertically aligning minipages
	
	I am a physicist and PhD candidate at the University of Milan, focusing on 
    quantum computing and hardware calibration.
    My work includes developing open-source tools like Qibocal and Qibolab 
    to make quantum hardware calibration more accessible. Outside of research, 
    I enjoy optimizing my computing setup with tools like Neovim and tmux, 
    fine-tuning configurations for a better experience and a smoother workflow.
\end{minipage}
\hfill % Automatic horizontal whitespace
\begin{minipage}[t]{0.5\textwidth} % Right column with the skills bar chart, change the width as needed
	\vspace{-\baselineskip} % Required for vertically aligning minipages
	
	\begin{barchart}{5.5} % The parameter to the barchart environment is the maximum width (in cm) of the longest bar
		\baritem{Python}{90}
		\baritem{C++}{60}
		\baritem{Bash}{60}
		\baritem{\LaTeX}{80}
		\baritem{Git}{90}
	\end{barchart}
\end{minipage}

% Output a series of bubbles showing your proficiency with environments and/or tools
\begin{center}
	\bubbles{6/Numpy, 6/Tensorflow, 4/Keras, 3/QuTiP} % Each bubble must be in the format of '<size>/<label>' and you can specify as many bubbles as will fit on the page
\end{center}

%----------------------------------------------------------------------------------------
%	EXPERIENCE
%----------------------------------------------------------------------------------------

\cvsect{Experience}

\begin{entrylist}
	\entry
		{2023 -- present}
		{Associate Researcher}
		{Technology Innovation Institute}
		{Contributing to quantum hardware calibration research and software development, 
        including tools like Qibocal and Qibolab for quantum systems.}
\end{entrylist}

%----------------------------------------------------------------------------------------
%	EDUCATION
%----------------------------------------------------------------------------------------

\cvsect{Education}

\begin{entrylist}
	\entry
		{2022 -- present}
		{PhD Candidate in Physics}
		{University of Milan}
		{Working on open-source quantum computing tools, focusing on superconducting
        chip calibration and quantum system software.}
	\entry
		{2020 -- 2022}
		{Master of Science in Physics}
		{University of Milan}
        {Advanced studies in theoretical and computational high energy physics.
        Grade 110/110 cum laude.}
	\entry
		{2017 -- 2020}
		{Bachelor of Science in Physics}
		{University of Milan}
		{Grade 110/110}
\end{entrylist}

\cvsect{Projects}
\begin{entrylist}
	\entry
		{}
        {\faGithub~\href{https://github.com/qiboteam/qibo}{\textbf{Qibo}}}
		{}
        {an open-source full stack API for quantum simulation and quantum hardware control.}

    \entry
        {}
        {\faGithub~\href{https://github.com/qiboteam/qibocal}{\textbf{Qibocal}}}
        {}
        {software providing Quantum Characterization Validation and Verification protocols. }

    \entry
        {}
        {\faGithub~\href{https://github.com/qiboteam/qibolab}{\textbf{Qibolab}}}
        {}
        {the dedicated Qibo backend for the automatic deployment of quantum circuits on quantum hardware. }

    \entry
        {}
        {\faGithub~\href{https://github.com/qiboteam/boostvqe}{\textbf{Boostvqe}}}
        {}
        {Boosting variational eigenstate preparation algorithms by double-bracket iteration.}

\end{entrylist}

\cvsect{Participation in Events}

\begin{entrylist}

    \entry
        {July 2023}
        {Summer School}
        {University of Trento, IT}
        {Summer school on Mathematical foundations of Quantum Machine Learning.}
        {}
         
    \entry
        {November 2023}
        {QTML}
        {CERN, Geneva, CH}
        {Quantum Techniques in Machine Learning.}
        {}

    \entry
        {January 2024}
        {QIP}
        {Taipei, TW}
        {Quantum Information Process.}
        {}

    \entry
        {March 2024}
        {March meeting}
        {Minneapolis, MN}
        {2024 APS March meeting.}
        {}

    \entry
        {May 2024}
        {ACAT24}
        {Stony Brook, NY}
        {22nd International Workshop on Advanced Computing and Analysis Techniques in Physics Research.}
        {}

    \entry
        {June 2024}
        {WQS24}
        {Copenhagen, DK}
        {Workshop on Quantum Software.}
        {}

    \entry
        {January 2025}
        {Quantum Technology Symposium.}
        {Abu Dhabi, UAE}
        {}
        {}
\end{entrylist}


\cvsect{Publications}

% \subsection{Journal Article(Accepted)}
% \cventry{2019}{\textbf{Pratik Dutta}, Sriparna Saha, Sanket Pai and Aviral Kumar}{}{Protein-protein Interaction based Generative Model for Improving Gene Clustering}{In \textit{\textbf{Scientific Reports-Nature}} (\textbf{Impact Factor: 4.12)}}{}

\textbf{Articles} \\

\begin{entrylist}
    \entry
        {2024}
        {Double-bracket quantum algorithms for high-fidelity ground state preparation}
        {}
        {on \href{https://doi.org/10.48550/arXiv.2408.03987}{\textbf{arXiv}} and under review.} 
    \entry
        {2024}
        {Qibocal: an open-source framework for calibration of self-hosted quantum devices.}
        {}
        {on \href{https://doi.org/10.48550/arXiv.2410.00101}{\textbf{arXiv}} and under review.} 
        {}
    \entry
        {2024}
        {Qibolab: an open-source hybrid quantum operating system}
        {}
        {on \href{https://quantum-journal.org/papers/q-2024-02-12-1247}{\textbf{Quantum Journal}}.} 
        {}
\end{entrylist}

\textbf{In conference Proceedings} \\

\begin{entrylist}

    \entry
        {2024}
        {Strategies for optimizing double-bracket quantum algorithms}
        {}
        {on \href{https://arxiv.org/abs/2408.07431}{\textbf{arXiv}}.} 
    \entry
        {2024}
        {An open-source framework for quantum hardware control}
        {}
        {on \href{https://arxiv.org/abs/2407.21737}{\textbf{arXiv}}.} 
    \entry
        {2024}
        {Beyond full statevector simulation with Qibo}
        {}
        {on \href{https://arxiv.org/abs/2408.00384}{\textbf{arXiv}}.} 
    \entry
        {2023}
        {Benchmarking machine learning models for quantum state classification}
        {}
        {on \href{https://arxiv.org/abs/2309.07679}{\textbf{arXiv}}.} 

\end{entrylist}
%----------------------------------------------------------------------------------------
%	ADDITIONAL INFORMATION
%----------------------------------------------------------------------------------------

\begin{minipage}[t]{0.3\textwidth} % Left column width
	\vspace{-\baselineskip} % Required for vertically aligning minipages

	\cvsect{Languages}
	
	\textbf{Italian} - Native\\
	\textbf{English} - Fluent\\
	\textbf{German} - Intermediate
\end{minipage}
\hfill % Automatic horizontal whitespace
\begin{minipage}[t]{0.3\textwidth} % Center column width
	\vspace{-\baselineskip} % Required for vertically aligning minipages
	
	\cvsect{Hobbies}
	
	Exploring advanced tools for computing, from Neovim to tmux, and optimizing my working environment for maximum productivity.
\end{minipage}
\hfill % Automatic horizontal whitespace
\begin{minipage}[t]{0.3\textwidth} % Right column width
	\vspace{-\baselineskip} % Required for vertically aligning minipages
	
	\cvsect{Non profit}
	
	Contributing to open-source quantum computing tools and research to make quantum technologies more accessible.
\end{minipage}

%----------------------------------------------------------------------------------------

\end{document}

